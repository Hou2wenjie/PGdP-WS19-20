\documentclass[handout, navsym]{tum-presentation}

\graphicspath{{figs/}}
\usepackage{fancybox}
\usepackage{listings}
\usepackage{xcolor}

\setbeamercovered{invisible}
\setbeamertemplate{navigation symbols}{}
\setbeamertemplate{section in toc}[sections numbered]
\setbeamertemplate{subsection in toc}[subsections numbered]
\setbeamertemplate{subsubsection in toc}[subsubsections numbered]
\numberwithin{equation}{section}

\definecolor{dkgreen}{rgb}{0,0.6,0}
\definecolor{gray}{rgb}{0.5,0.5,0.5}
\definecolor{mauve}{rgb}{0.58,0,0.82}

\lstset{
	basicstyle=\footnotesize\ttfamily,% 基本风格
         numbers=left,    % 行号
         numberstyle=\tiny,
         numbersep=10pt,  % 行号间隔 
         tabsize=2,       % 缩进
         extendedchars=true, % 扩展符号?
         breaklines=true, % 自动换行
         language=C++,
         frame=shadowbox,  % 框架左边竖线
         xleftmargin=19pt,% 竖线左边间距
         showspaces=false,% 空格字符加下划线
         showstringspaces=false,% 字符串中的空格加下划线
         showtabs=false,  % 字符串中的tab加下划线
       xleftmargin=2em,xrightmargin=2em,aboveskip=1em
}

\title{Praktikum: Grundlagen der Programmierung}
\author[Wenjie Hou]{Wenjie Hou}
\date{3. Tutorübung}
\institute{\theuniversity\par Fakultät für Informatik}
\footline{\insertshortauthor~|~\insertshorttitle~|~3. Tutorübung}

\begin{document}

\begin{frame}[noframenumbering]
  \titlepage
\end{frame}

\section{Lecture Review}

\section{P01: Syntaxbaum}
\begin{frame}[fragile]
  \frametitle{P01: Syntaxbaum}
  \vspace*{\fill}
\large  For the following MiniJava program, draw the syntax tree from the lecture according to the MiniJava grammar. \par
~\\
\bigskip
\center
\begin{lstlisting}
int x,r;
int n;
r = 1;
n = 1;
x = readInt();
while (n < x) {
    if (r % 1 == 0)
        r = r * n;
    else {
        r = r * (-n);
    }
    n = n + 1;
    write (r);
}
\end{lstlisting}

\vspace*{\fill}
\end{frame}

\section{P02: Binäre Zahlen}

\begin{frame}[fragile]
\frametitle{P02: Binäre Zahlen}
\framesubtitle{Regular Expression}
\vspace*{\fill} \large
\begin{tabular}{l|l}
Metacharacter&Description\\
$?$&Matches the preceding element zero or one time.\\
$*$&Matches the preceding element zero or more times.\\
$|$&The choice (also known as alternation or set union) operator matches either the expression\\ &before or the expression after the operator.\\
$\{m,n\}$&Matches the preceding element at least m and not more than n times.\\
\end{tabular}
\vspace*{\fill}

\end{frame}

\begin{frame}[fragile]
\frametitle{P02: Binäre Zahlen}
\framesubtitle{Regular Expression}
\vspace*{\fill} \large
Problem statement: See Artemis \url{https://artemis.ase.in.tum.de/overview/37/exercises/789}\\
\vspace*{\fill}
\end{frame}

\begin{frame}[fragile]
\frametitle{P02: Binäre Zahlen}
\framesubtitle{Regular Expression}
\vspace*{\fill} \large
Problem statement: See Artemis \url{https://artemis.ase.in.tum.de/overview/37/exercises/789}\\
\bigskip
Solution: $0(b|B)(0|1)((\_|0|1)*(0|1))?$\\
\vspace*{\fill}
\end{frame}

\begin{frame}[fragile]
\frametitle{P02: Binäre Zahlen}
\framesubtitle{Regular Expression}
\vspace*{\fill} \large
Problem statement: See Artemis \url{https://artemis.ase.in.tum.de/overview/37/exercises/789}\\
\bigskip
Solution: $0(b|B)(0|1)((\_|0|1)*(0|1))?$\\
\bigskip
Or: $0(b|B)(\_(0|1)\{4\})*(0|1)\{1,3\}$
\vspace*{\fill}
\end{frame}

\begin{frame}[fragile]
\frametitle{P03: Palindrome}
\vspace*{\fill} \big

In this task you should write a program which checks for a number whether the number is a palindrome. A number is a palindrome if it represents the same value read forwards and backwards.\par
~\\
First the user should be asked for a positive number. Repeat the quert until the user actually enters a positive number. The number entered is then to be converted into an array of digits on which the palindrome property is finally checked. Then either print \textsl{ \color{blue}"palindrome"} or textsl{ \color{blue}"Kein Palindrom"}.\par
~\\
The number is to be read by calling  textsl{ \color{red}readInt()}. It must not be converted into a string at any time. Only use MiniJava methods. In particular, do not use the method Math.log10(double a).\par
\bigskip
\textbf{\large Example Output:}
\center
\begin{lstlisting}
<Geben Sie eine Zahl n >= 0 ein.
>1221
<Palindrom
<Geben Sie eine Zahl n >= 0 ein.
>123
<Kein Palindrom
\end{lstlisting}
\vspace*{\fill}
\end{frame}

\subsection{Code}
\begin{frame}[fragile]
\frametitle{Code - 1}
\vspace*{\fill}
\begin{lstlisting}
 public static void main(String[] args) {
        int n = read("Geben Sie eine Zahl n >= 0 ein.");
        while (n < 0) {
          n = read("Geben Sie eine Zahl n >= 0 ein.");
        }

        // We first count the number of digits of the number n by dividing the number by 10 
        //until it has the value 0. The number of divisions corresponds to the number of digits.
        int numberOfDigits = 0;
        int t = n;
        while (t != 0) {
          numberOfDigits++;
          t = t / 10;
        }

        int[] digits = new int[numberOfDigits];

\end{lstlisting}
\vspace*{\fill}
\end{frame}

\begin{frame}[fragile]
\frametitle{Code - 2}
\vspace*{\fill}
\begin{lstlisting}
	
	// We now read the digits into an array. We get a digit as division remainder by 10.
	//The order in which we place the number in the array does 
	//not matter for the palindrome test.
        int i = 0;
        while (n != 0) {
          int digit = n % 10;
          digits[i] = digit;
          n = n / 10;
          i++;
        }

\end{lstlisting}
\vspace*{\fill}
\end{frame}

\begin{frame}[fragile]
\frametitle{Code - 3}
\vspace*{\fill}
\begin{lstlisting}
        int notMatching = 0;
        i = 0;
        while (i < numberOfDigits / 2) {
          if (digits[i] != digits[numberOfDigits - i - 1])
            notMatching++;
          i++;
        }

        if (notMatching == 0)
          write("Palindrom");
        else
          write("Kein Palindrom");
      }


\end{lstlisting}
\vspace*{\fill}
\end{frame}

 
\begin{frame}[fragile]
\frametitle{P04: Pascalsches Dreieck}
\vspace*{\fill} \large

The Pascal triangle is built step by step, starting with line 0. To do this, calculate the nth line from the $(n - 1)^{th}$ line as follows:\par
\begin{itemize}
	\item The number of elements of line $n$ is $n + 1$.
	\item The first and last number of each line is always 1.
	\item The $i^{th}$ element of line $n$ corresponds to the sum of the $i^{th}$ and $(i-1)^{th}$ elements of line $(n - 1)$.
\end{itemize}
Write a Java method calls textsl{ \color{red}int[][] pascalDreieck(int n)} , which calculates the Pascal triangle. The parameter $n$ specifies the number of lines to be calculated; an array containing the Pascal triangle is returned. Also implement the main method, in which a Pascal triangle for a user-defined size n is output. Assume that the input is $n \ge 0$.\par
\vspace*{\fill}
\end{frame}

\begin{frame}[fragile]
\frametitle{P04: Pascalsches Dreieck}
\vspace*{\fill}
\textbf{\large Example Output:}
\begin{lstlisting}
<Gib die Zeilenzahl an:
>5
<n=0    1
<n=1    1   1
<n=2    1   2   1
<n=3    1   3   3   1
<n=4    1   4   6   4   1
\end{lstlisting}

\vspace*{\fill}
\end{frame}

\subsection{Code}
\begin{frame}[fragile]
\frametitle{Code}
\vspace*{\fill}
\center
\begin{lstlisting}
// Funktion zur Berechnung der ersten n Zeilen des Pascalschen Dreiecks.
    public static int[][] pascalDreieck(int n) {
        int[][] dreieck = new int[n][];
        for (int m = 0; m < n; m++) {
            dreieck[m] = new int[m + 1];
            dreieck[m][0] = 1;
            dreieck[m][m] = 1;
            for (int i = 1; i < m; i++)
                dreieck[m][i] = dreieck[m - 1][i - 1] + dreieck[m - 1][i];
        }
        return dreieck;
    }

\end{lstlisting}
\vspace*{\fill}
\end{frame}

\begin{frame}[fragile]
\frametitle{Code}
\vspace*{\fill}
\center
\begin{lstlisting}
    public static void main(String[] args) {
        int zeilenzahl = read("Gib die Zeilenzahl an:");
        int[][] dreieck = pascalDreieck(zeilenzahl);
        for (int i = 0; i < zeilenzahl; i++) {
            writeConsole("n=" + i);
            for (int j = 0; j < dreieck[i].length; j++) {
                writeConsole("\t");
                writeConsole(dreieck[i][j]);
            }
            writeLineConsole();
        }
    }

\end{lstlisting}
\vspace*{\fill}
\end{frame}
\begin{frame}[fragile]
\vspace*{\fill} 
\begin{center}
 \Huge Thank You!
\end{center}
\vspace*{\fill} 
\end{frame}




\end{document}
